%%%%%%%%%%%%%%%%%%%%%%%%%%%%%%%%%%%%%%%%%
% University Assignment Title Page 
% LaTeX Template
% Version 1.0 (27/12/12)
%
% This template has been downloaded from:
% http://www.LaTeXTemplates.com
%
% Original author:
% WikiBooks (http://en.wikibooks.org/wiki/LaTeX/Title_Creation)
%
% License:
% CC BY-NC-SA 3.0 (http://creativecommons.org/licenses/by-nc-sa/3.0/)
% 
% Instructions for using this template:
% This title page is capable of being compiled as is. This is not useful for 
% including it in another document. To do this, you have two options: 
%
% 1) Copy/paste everything between \begin{document} and \end{document} 
% starting at \begin{titlepage} and paste this into another LaTeX file where you 
% want your title page.
% OR
% 2) Remove everything outside the \begin{titlepage} and \end{titlepage} and 
% move this file to the same directory as the LaTeX file you wish to add it to. 
% Then add \input{./title_page_1.tex} to your LaTeX file where you want your
% title page.
%
%%%%%%%%%%%%%%%%%%%%%%%%%%%%%%%%%%%%%%%%%
%\title{Title page with logo}
%----------------------------------------------------------------------------------------
%	PACKAGES AND OTHER DOCUMENT CONFIGURATIONS
%----------------------------------------------------------------------------------------

\documentclass[12pt]{article}
\usepackage[portuguese]{babel}
\usepackage[utf8x]{inputenc}
\usepackage{amsmath}
\usepackage{graphicx}
\usepackage[colorinlistoftodos]{todonotes}

\begin{document}

\begin{titlepage}

\newcommand{\HRule}{\rule{\linewidth}{0.5mm}} % Defines a new command for the horizontal lines, change thickness here

\center % Center everything on the page
 
%----------------------------------------------------------------------------------------
%	HEADING SECTIONS
%----------------------------------------------------------------------------------------

\textsc{\LARGE Universidade Federal da Bahia}\\[1.5cm] 
\textsc{\Large ENGA73 - Sistemas Robóticos}\\[0.5cm] 

%----------------------------------------------------------------------------------------
%	TITLE SECTION
%----------------------------------------------------------------------------------------

\HRule \\[0.4cm]
{ \huge \bfseries Cinemática e planejamento de trajetória do UR5}\\[0.4cm] % Title of your document
\HRule \\[1.5cm]
 
%----------------------------------------------------------------------------------------
%	AUTHOR SECTION
%----------------------------------------------------------------------------------------

\begin{minipage}{0.4\textwidth}
\begin{flushleft} \large
\emph{Discentes:} \\
{\normalsize
\hspace{1em} Matheus Menezes \\
\hspace{1em} Rafael Magalhẽs \\
\hspace{1em} Yuri Oliveira}
\end{flushleft}
\end{minipage}
~
\begin{minipage}{0.4\textwidth}
\begin{flushright} \large
\emph{Docente:} \\
André Scolari
\end{flushright}
\end{minipage}\\[2cm]

%----------------------------------------------------------------------------------------
%	DATE SECTION
%----------------------------------------------------------------------------------------

{\large \today}\\[2cm] % Date, change the \today to a set date if you want to be precise

%----------------------------------------------------------------------------------------
%	LOGO SECTION
%----------------------------------------------------------------------------------------

\includegraphics[scale=0.3]{images/ufba_logo.jpg}\\[1cm] % Include a department/university logo - this will require the graphicx package
 
%----------------------------------------------------------------------------------------

\vfill % Fill the rest of the page with whitespace

\end{titlepage}


\section{Introdução}

O principal objetivo desse trabalho é simular o manipulador robótico UR5
da \textit{universal robots} em uma missão de \textit{pick and place} de 
uma lata. Para isso, se utilizou a \textit{framework} de robótica ROS 
e o simulador gazebo. O modelo do robô utilizado é o próprio modelo da 
farbicante, porém todo o código de cinemática, planejamento de trajetória e
de missão foram desenvolvidos pela equipe deste trabalho.

O desenvolvimento do projeto pode ser dividído em quatro partes, desenvolvimento de 
cinemática direta, cinemática inversa, planejamento de trajetória e planejamento de 
missão. A cinemática inversa foi utilizada para a calcular a posição de cada junta 
para se alcançar as poses definidas para operação de \textit{pick and place}. O planejamento
de trajetória foi desenvolvido para cálculo da posição de cada junta em cada instante de tempo
da missão, de forma que o robô se movesse de forma suave. A cinemática direta foi desenvolvida
somente para fins de estudo da modelagem do robô. Por fim, o planejamento da missão foi realizado
para a integração de todos os sistemas na execução da tarefa.

\section{Cinemática Direta}

\section{Cinemática Inversa}

\section{Planejamento de Trajetória}

\section{Planejamento de Missão}

\end{document}